%%================================================
%% Chapter 1
%%================================================
\chapter{บทนำ}
%\label{intro}
\label{chapter1}

\section{ที่มาและความสำคัญ}
    ข้อดีและข้อเสียของอินเทอร์เน็ต มีมากมาย เช่น เพื่อการทําธุรกรรมทางการเงิน ส่งผล\mbox{ให้}การซื้อขายง่ายขึ้น แต่มีความเสี่ยงในการถูกหลอกหลวงได้ง่ายด้วยเช่นกัน การใช้งานอินเทอร์เน็ตเพื่อการศึกษาก็มีข้อเสีย เช่น การที่ผู้สอนขาดปฏิสัมพันธ์กับผู้เรียนโดยตรง หรือการที่ถูกสิ่งเร้าจากภายนอกคอย\mbox{รบกวน} จากสาเหตุข้างต้น จึงเป็นที่มาของโครงงาน REST API สําหรับการส่งงานเขียนโปรแกรมด้วย OpenAPI (REST API for Programming assignment With OpenAPI) ที่จะยกระดับการทําการบ้านหรือการทําข้อสอบประเภทการเขียนโปรแกรมในรูปแบบออนไลน์ ให้มีความปลอดภัยมากขึ้น และสามารถตรวจสอบความถูกต้องได้ รวมถึงเป็นการอํานวยความสะดวกให้แก่อาจารย์หรือผู้สอนในการตรวจข้อสอบ ป้องกันไม่ให้เกิดการตรวจข้อสอบที่ผิดพลาด และสามารถลดเวลาในการตรวจข้อสอบได้เป็นอย่างมาก

\section{วัตถุประสงค์}
\begin{enumerate}
    \item เพื่อพัฒนา API ตามแนวทางของ REST API สําหรับการส่งงานเขียนโปรแกรม
    \item เพื่อพัฒนาระบบสําหรับรองรับการเรียกใช้ API ในการการส่งงานเขียนโปรแกรม
    \item เพื่อเพิ่มความรวดเร็วในการส่งงานหรือส่งข้อสอบการเขียนโปรแกรม
    \item เพื่ออํานวยความสะดวกให้ผู้สอนสําหรับการตรวจงานการเขียนโปรแกรม
    \item เพื่อเพิ่มความแม่นยําและถูกต้องในการตรวจงานการเขียนโปรแกรม
\end{enumerate}


\section{ขอบเขตการดำเนินงาน}
\begin{enumerate}
    \item พัฒนา API ตามแนวทางของ REST API สําหรับการส่งงานเขียนโปรแกรม โดยจะพัฒนา API 2 รายการดังนี้
    \begin{enumerate}[\theenumi.\arabic*]
        \item API รายการแรกเป็น Compile API ทําหน้าที่ตรวจสอบโปรแกรม และส่งผลลัพธ์ของโปรแกรมกลับมาเป็น output
        \item API รายการที่สองเป็น Problem API  ทําหน้าที่รับข้อมูลจากผู้ใช้และทําการตรวจสอบโปรแกรม โดยทําการเปรียบเทียบกับผลเฉลย และส่งผลลัพธ์กลับเป็น output
    \end{enumerate}        
    \item API ที่พัฒนาทั้ง 2 รายการ จะรับข้อมูลคําตอบได้เฉพาะไฟล์ที่เป็นภาษาคอมพิวเตอร์
\end{enumerate}
\section{ขั้นตอนการดำเนินงาน}
\begin{enumerate}
    \item เริ่มศึกษาค้นคว้าเกี่ยวกับเรื่องที่ผู้จัดทำต้องใช้ในโครงงาน
    \item กำหนดขอบเขตที่จะทำหลังจากสรุปแนวทางที่ได้จากอาจารย์ที่ปรึกษา
    \item พูดคุยกับผู้ร่วมโครงงานเกี่ยวกับการแบ่งหน้าที่ และ จัดแบ่งภาระงานอย่างเหมาะสม
    \item ลงมือทำโครงงานตามที่ได้รับมอบหมาย
    \item ทดสอบการใช้งานของ API
    \item จัดทำรายงาน
    \item นำเสนอโครงงาน
\end{enumerate}

\section{ผลที่คาดว่าจะได้รับ}
ผู้จัดทำโครงงานคาดหวังว่า ผู้ใช้จะได้รับประโยชน์การใช้ service เพื่อแบ่งเบาภาระ\mbox{หน้าที่}ในการตรวจข้อสอบ ลดเวลาในส่วนการตรวจซึ่งไม่จำเป็น และสำหรับผู้ที่เป็นนักศึกษาจะสามารถมีช่องทางการส่งข้อสอบได้อย่างสะดวก และทราบผลได้อย่างรวดเร็วยิ่งขึ้น
\begin{landscape}

\section{ตารางการดำเนินงาน}

    \begin{table}[h]
        \centering
        
        \begin{tabular}{|*{21}{c|}}
        \hline 
        \multirow {2}{*}{หัวข้อ} & \multicolumn{4}{c|}{สิงหาคม} & \multicolumn{4}{c|}{กันยายน} & \multicolumn{4}{c|}{ตุลาคม} & \multicolumn{4}{c|}{พฤศจิกายน} & \multicolumn{4}{c|}{ธันวาคม} \\\cline {2-21}
        & 1 & 2 & 3 & 4 & 1 & 2 & 3 & 4 &  1 & 2 & 3 & 4 & 1 & 2 & 3 & 4 & 1 & 2 & 3 & 4  \\\hline 
         สรุปแนวทางการจัดทำภายในคณะผู้จัดทำ & x & x &  &  &  &  &  &  &  &  &  &  &    &  &  &  &  &  &  &  \\\hline 
         จัดทำรายงานข้อเสนอโครงงาน &  &  & x &  &  &  &  &  &  &  &  &  &  &  &  &  &  &  &  &    \\\hline 
         นำส่งรายงาน และปรับแก้รายงานกับอาจารย์ที่ปรึกษา &  &  &  & x  &  &  &  &  &  &  &  &  &  &  &  &  &  &  &  &  \\\hline 
         จัดทำรายงานการค้นคว้าเบื้องต้น &  &  &  &  & x & x  & x  & x  &  &  &  &  &  &  &  &  &  &  &  & \\\hline 
        จัดทำโครงงานในส่วนอื่น ๆ &  &  &  &  &  &  &  &  & x & x & x & x &  &  &  &  &  &  &  &   \\\hline 
        จัดทำรายงานความคืบหน้าและสอบครั้งที่ 1 &  &  &  &  &  &  &  &  &  &  &  &  & x & x & x & x &  &  &  &   \\\hline 
         อัปเดตความคืบหน้า &  &  &  &  &  &  &  &  &  &  &  &  &  &  &  &  & x & x & x & x  \\\hline 
        
        \end{tabular}
        \caption{การดำเนินโครงงาน}
        \label{tab:my_label}
        \vspace{0.5em}
    
        \begin{tabular}{|*{21}{c|}}
        \hline 
         \multirow {2}{*}{หัวข้อ} & \multicolumn{4}{c|}{มกราคม} & \multicolumn{4}{c|}{กุมภาพันธ์} & \multicolumn{4}{c|}{มีนาคม} & \multicolumn{4}{c|}{เมษายน} & \multicolumn{4}{c|}{พฤษภาคม} \\\cline {2-21}
        & 1 & 2 & 3 & 4 & 1 & 2 & 3 & 4 &  1 & 2 & 3 & 4 & 1 & 2 & 3 & 4 & 1 & 2 & 3 & 4  \\\hline 
         อัปเดตความคืบหน้า จัดทำรายงานความคืบหน้าโครงงานครั้งที่ 2  & x & x & x & x & x & x &  &  &  &  &  &  &    &  &  &  &  &  &  &  \\\hline 
         ทดสอบการใช้งานของ  &  &  &  &  &  &  & x & x &  &  &  &  &  &  &  &  &  &  &  &    \\\hline 
         จัดทำโครงงานให้แล้วเสร็จ  &  &  &  &  &  &  &  &  & x & x & x & x &  &  &  &  &  &  &  &  \\\hline 
         ตรวจสอบความเรียบร้อยและจัดทำปริญญานิพนธ์และบทความ &  &  &  &  &  &   &   &   &  &  &  &  & x & x & x & x &  &  &  & \\\hline 
        สอบโครงงานครั้งที่ 2 และส่งปริญญานิพนธ์ฉบับสมบูรณ์ &  &  &  &  &  &  &  &  &  &  &  &  &  &  &  &  & x &  &  &   \\\hline 
    
        \end{tabular}
        \caption{การดำเนินโครงงาน (ต่อ)}
        \label{tab:my_label}
    \end{table}


    



\end{landscape}
